%%%%%%%%%%%%%%%%%%%%%%%%%%%%%%%%%%%%%%%%%
% Thin Sectioned Essay
% LaTeX Template
% Version 1.0 (3/8/13)
%
% This template has been downloaded from:
% http://www.LaTeXTemplates.com
%
% Original Author:
% Nicolas Diaz (nsdiaz@uc.cl) with extensive modifications by:
% Vel (vel@latextemplates.com)
%
% License:
% CC BY-NC-SA 3.0 (http://creativecommons.org/licenses/by-nc-sa/3.0/)
%
%%%%%%%%%%%%%%%%%%%%%%%%%%%%%%%%%%%%%%%%%

%----------------------------------------------------------------------------------------
%	PACKAGES AND OTHER DOCUMENT CONFIGURATIONS
%----------------------------------------------------------------------------------------

\documentclass[a4paper, 11pt]{article} % Font size (can be 10pt, 11pt or 12pt) and paper size (remove a4paper for US letter paper)

\usepackage[protrusion=true,expansion=true]{microtype} % Better typography
\usepackage[frenchb]{babel}
\usepackage{graphicx} % Required for including pictures
\usepackage{wrapfig} % Allows in-line images

\usepackage{mathpazo} % Use the Palatino font
\usepackage[utf8]{inputenc} % UTF-8 encoding for input, to get french special characters recognized
\usepackage[T1]{fontenc} % Required for accented characters
\linespread{1.05} % Change line spacing here, Palatino benefits from a slight increase by default

\makeatletter
\renewcommand\@biblabel[1]{\textbf{#1.}} % Change the square brackets for each bibliography item from '[1]' to '1.'
\renewcommand{\@listI}{\itemsep=0pt} % Reduce the space between items in the itemize and enumerate environments and the bibliography

\renewcommand{\maketitle}{ % Customize the title - do not edit title and author name here, see the TITLE block below
\begin{flushright} % Right align
{\LARGE\@title} % Increase the font size of the title

\vspace{50pt} % Some vertical space between the title and author name

{\large\@author} % Author name
\\\@date % Date

\vspace{40pt} % Some vertical space between the author block and abstract
\end{flushright}
}

%----------------------------------------------------------------------------------------
%	TITLE
%----------------------------------------------------------------------------------------

\title{\textbf{Techniques d'optimisation des temps de chargements dans les jeux à monde ouvert}\\ % Title
Ou comment réduire les temps de chargement} % Subtitle

\author{\textsc{Christian NGO \& Jonathan MULLER} % Author
\\{\textit{ESGI - 5ème année IJV}}} % Institution

\date{\today} % Date

%----------------------------------------------------------------------------------------

\begin{document}

\maketitle % Print the title section

%----------------------------------------------------------------------------------------
%	ABSTRACT AND KEYWORDS
%----------------------------------------------------------------------------------------

%\renewcommand{\abstractname}{Summary} % Uncomment to change the name of the abstract to something else

\begin{abstract}
Abstract place-holder
\end{abstract}

\hspace*{3,6mm}\textit{Mots clefs:} chargement , optimisation , ressources , streaming , jeu % Keywords

\vspace{30pt} % Some vertical space between the abstract and first section

%----------------------------------------------------------------------------------------
%	ESSAY BODY
%----------------------------------------------------------------------------------------

\section*{Introduction}

Les jeux vidéos comportent année après année un nombre grandissant de ressources necessaire pour leur fonctionnement. Des textures aux modèles 3D, le poids de ces ressources augmente tout comme leur nombre, l'augmentation de la taillé mémoire des machines permettant aux créateurs de jeux de charger plus de ressources à la fois. Cependant, augmenter la taille et le nombre des ressources à un prix, celui du temps de chargement. Dans ce mémoire nous verrons comment réduire ces temps de chargement en s'adaptant aux médias qui permettent de stocker les ressources du jeu, en tirant partie des capacités des processeurs, de l'espace disponible et de la mémoire RAM, des différents formats de compression, mais aussi des adaptations de gameplay qui peuvent permettre un chargement transparant pour l'utilisateur.

%------------------------------------------------

\section*{Comment influencer les temps de chargement}

Les temps de chargement peuvent être influencés de différentes manières.

Nous pouvons améliorer les temps de chargement en utilisant des techniques de stockage qui varient en fonction du média sur lequel le jeu est enregistré. Par exemple, les temps d'accès étant plutôt long sur un cd-rom, il faut veiller à limiter le nombre d'accès aux ressources et penser à charger des elements qui ne seront pas forcèment affichés tout de suite, lors des premiers chargements si la mémoire le permet.
Au contraire sur un disque SSD, les temps d'accès étant très bas et la capacité de transfert élevée, on pourra séparer les ressources dans des fichiers différents afin de les charger quasiement à la volée.

Les temps de chargement peuvent égallement être réduits en utilisant le processeur pour décompresser des ressources: ainsi le temps necessaire au transfert de la donnée depuis le disque à la mémoire est réduit puisque la donnée est compressée sur le dit disque. Il faut cependant que le mode de compression soit adapté à la puissance du processeur et à la vitesse de transmission du support de stockage.

En modifiant l'algorithme du jeu, nous pouvons aussi influer sur le chargement des données, en évitant par exemple de stocker plusieurs fois des ressources très similaires. Si nous prennons l'exemple des sprites 2D, nous pouvons stocker l'image de base une fois, puis effectuer toutes les déclinaisons de couleurs de manière algorithmique en modifiant la palette de couleurs. De cette façon, nous n'avons qu'à stocker une texture, puis uniquement des palettes de couleurs.

\begin{wrapfigure}{l}{0.4\textwidth} % Inline image example
\begin{center}
%\includegraphics[width=0.38\textwidth]{fish.png}
\end{center}
\caption{Fish}
\end{wrapfigure}
Aliquam fringilla non diam sed varius. Suspendisse tellus felis, hendrerit non bibendum ut, adipiscing vitae diam. Lorem ipsum dolor sit amet, consectetur adipiscing elit. Nulla lobortis purus eget nisl scelerisque, commodo rhoncus lacus porta. Vestibulum vitae turpis tincidunt, varius dolor in, dictum lectus. Aenean ac ornare augue, ac facilisis purus. Sed leo lorem, molestie sit amet fermentum id, suscipit ut sem. Vestibulum orci arcu, vehicula sed tortor id, ornare dapibus lorem. Praesent aliquet iaculis lacus nec fermentum. Morbi eleifend blandit dolor, pharetra hendrerit neque ornare vel. Nulla ornare, nisl eget imperdiet ornare, libero enim interdum mi, ut lobortis quam velit bibendum nibh.

Morbi tempor congue porta. Proin semper, leo vitae faucibus dictum, metus mauris lacinia lorem, ac congue leo felis eu turpis. Sed nec nunc pellentesque, gravida eros at, porttitor ipsum. Praesent consequat urna a lacus lobortis ultrices eget ac metus. In tempus hendrerit rhoncus. Mauris dignissim turpis id sollicitudin lacinia. Praesent libero tellus, fringilla nec ullamcorper at, ultrices id nulla. Phasellus placerat a tellus a malesuada.

%------------------------------------------------

\section*{Conclusion}

Fusce in nibh augue. Cum sociis natoque penatibus et magnis dis parturient montes, nascetur ridiculus mus. In dictum accumsan sapien, ut hendrerit nisi. Phasellus ut nulla mauris. Phasellus sagittis nec odio sed posuere. Vestibulum porttitor dolor quis suscipit bibendum. Mauris risus lectus, cursus vitae hendrerit posuere, congue ac est. Suspendisse commodo eu eros non cursus. Mauris ultrices venenatis dolor, sed aliquet odio tempor pellentesque. Duis ultricies, mauris id lobortis vulputate, tellus turpis eleifend elit, in gravida leo tortor ultricies est. Maecenas vitae ipsum at dui sodales condimentum a quis dui. Nam mi sapien, lobortis ac blandit eget, dignissim quis nunc.

\begin{enumerate}
\item First numbered list item
\item Second numbered list item
\end{enumerate}

Donec luctus tincidunt mauris, non ultrices ligula aliquam id. Sed varius, magna a faucibus congue, arcu tellus pellentesque nisl, vel laoreet magna eros et magna. Vivamus lobortis elit eu dignissim ultrices. Fusce erat nulla, ornare at dolor quis, rhoncus venenatis velit. Donec sed elit mi. Sed semper tellus a convallis viverra. Maecenas mi lorem, placerat sit amet sem quis, adipiscing tincidunt turpis. Cras a urna et tellus dictum eleifend. Fusce dignissim lectus risus, in bibendum tortor lacinia interdum.

%----------------------------------------------------------------------------------------
%	BIBLIOGRAPHY
%----------------------------------------------------------------------------------------

\bibliographystyle{unsrt}

\bibliography{References}

%----------------------------------------------------------------------------------------

\end{document}
